\chapter{Implementace připojení ke vzdálené ploše}
\label{chap:implementace-guacamole}
Implementace připojení ke vzdálené ploše probíhala v několika fázích. Nejdříve bylo potřeba spustit řešení \emph{Guacamole} lokálně a ujistit se, že je použitelné pro potřeby tohoto projektu. Bylo také nutné zanalyzovat, jaké možnosti poskytuje a které vlastnosti jsou potřebné a které naopak přebytečné (správa uživatelů, zadávání parametrů připojení a jejich skupin). Dále je implementována podpora klávesnice na obrazovce, ukládání snímků obrazovky a mnoho dalších rozšířených možností, které nebyly do základní verze vyžadovány a byly spíše kontraproduktivní s ohledem na zvýšení komplexnosti kódu a ztížení odladění celého řešení. 

Jakmile byla funkčnost ověřena, klientská část řešení byla přesunuta do modulu pro \emph{Drupal} a upravena pro tamnější podmínky. Bylo nutné změnit \emph{HTML} výstup, kdy \emph{Drupal} si v základu buduje strukturu pomocí několika šablonových souborů pro stránku a její elementy (viz. kapitola \ref{sec:tema-vzhledu}). Dále byly přidány a upraveny JavaScriptové knihovny z \emph{Guacamole}, ze kterých byl odstraněn přebytečný kód. 

Nástroj \emph{Guacamole} poskytuje mnohá nastavení a pro ulehčení byla do modulu přidána administrační stránka, která přímo zapisuje do konfiguračního souboru a tím eliminuje potřebu uživatelského připojení na server a manuální konfigurace. 

\section{Architektura připojení ke vzdálené ploše}
Základně poskytuje \emph{Guacamole} komplexní řešení na propojení ke vzdálené ploše včetně stránky pro management uživatelů, kde si každý z nich může vytvořit své vlastní připojení, u kterých se zobrazuje náhled v posledním momentu připojení a podobně. Jak je zobrazeno na diagramu \ref{fig:arch_guacamole}, součástí \emph{Guacamole} je i webová aplikace v jazyce \emph{HTML}, která implementuje knihovnu Guacamole-js a tím i komunikaci s \emph{JAVA} \gls{servlet}em. Pro potřeby tohoto projektu bylo nutné tyto JavaScriptové knihovny převzít a implementovat do modulu pro \emph{Drupal}. 

\begin{figure}[htp] 
  \centering{\includegraphics[width=12cm]{img/architecture-guacamole-crop.pdf}}
  \caption{Neupravená architektura a rozčlenění řešení}
  \label{fig:arch_guacamole}
\end{figure}  

Díky skutečnosti, že \emph{JAVA} Server a webová aplikace nemusí nutně běžet na stejném serveru, jak by tomu bylo při základním nastavení, bylo nutné změnit architekturu \emph{AJAX} požadavků. Ty totiž musejí v typickém případě být zasílány pouze na stejnou doménu kvůli prevenci \gls{xss}. Pro vyřešení tohoto problému byla změněna implementace na \gls{cors} požadavky, k čemuž musel být uzpůsoben i \emph{JAVA} \gls{servlet} kontejner, u kterého byly povoleny požadavky z dané domény. 

Uživatel proto pracuje základně pouze s portálem postaveném na \emph{Drupalu}, do kterého byl přidán modul implementující stránku určenou ke komunikaci s \emph{Guacamole} Servlet kontejnerem za pomoci upravené JavaScriptové knihovny Guacamole-js, jak je vyobrazeno na diagramu \ref{fig:arch_drupal}.

\begin{figure}[htp] 
  \centering{\includegraphics[width=12cm]{img/architecture-drupal-crop.pdf}}
  \caption{Architektura přizpůsobená pro portál ESF}
  \label{fig:arch_drupal}
\end{figure}  

\section{Změna autentizace uživatelů}
\emph{Guacamole} poskytuje \gls{api} pro správu uživatelských účtů a definic připojení. Jsou implementovány dvě metody, jedna spoléhá na uložení dat do konfiguračního souboru, druhá pracuje s \emph{MySQL} databází. Obě rozšiřují třídu \emph{SimpleAuthenticationProvider}, ke které přidávají potřebnou funkcionalitu. 

Tento projekt vyžaduje pouze jedinou definici připojení a pro jednotlivé uživatele se mění jen jejich přihlašovací jméno. Proto byla vytvořena nová třída \emph{DrupalAuthenticationProvider}, která načítá adresu serveru a port přímo z konfiguračního souboru a aplikuje je pro všechna připojení.

Pro nastavení přihlašovacích údajů je potřeba se nejdříve k serveru přihlásit. Pro minimalizaci úkonů potřebných ke každému připojení ke vzdálené ploše a také k zjednodušení celého procesu si musí každý uživatel portálu uložit své přihlašovací údaje do uživatelského profilu na portálu. Odsud jsou načteny při přihlášení, ke kterému je použito dvou serverových služeb - \emph{connect} a \emph{login}. Proces je znázorněn na diagramu \ref{fig:login_process}. 

Komunikace je započata přímo v klientském prohlížeči z JavaScriptu. Po přístupu na stránku \texttt{/aspi} je nejdříve kontaktována pomocí technologie \gls{ajax} stránka \texttt{/aspi/ajax} na portálu, která se připojí ke službě Login. \emph{Guacamole} \gls{servlet} si vytvoří nové sezení a jeho identifikátor odešle zpět \emph{PHP} kódu. Pokud vše proběhne v pořádku, je identifikátor \gls{session} odeslán pomocí \texttt{set-cookie} hlavičky zpět stránce \emph{/aspi}, kde si informaci uloží prohlížeč jako cookie pro pozdější komunikaci. 

V případě jakékoliv chyby je chybové hlášení vypsáno na obrazovce a zalogováno v systému bez jakýchkoliv dalších pokusů o připojení, uživatel je tedy nucen buď stránku obnovit, nebo kontaktovat správce stránek. Pokud však vše proběhne v pořádku, je inicializována smyčka přímých volání služby \emph{tunnel} poskytované Guacamole \gls{servlet} kontejnerem za pomocí technologie \gls{cors}. Komunikace probíhá za pomocí vlastního protokolu, který je konvertován na grafický výstup na obrazovce uživatele. Ten je od této chvíle schopen pomocí myši a klávesnice ovládat vzdálenou plochu a tím i \gls{aspi}.

\begin{figure}[]
  \includegraphics[width=12cm]{img/login-process-crop.pdf}
  \caption{Proces připojení ke vzdálené ploše}
  \label{fig:login_process}
\end{figure}  

\section{Přesun JavaScript kódu a CSS stylů do modulu Drupalu}
Pro ulehčení správy \gls{session}, která by jinak musela být synchronizována mezi dvěma stránkami, byl veškerý kód stránky z \emph{Guacamole JAVA} projektu přesunut do modulu esf\_module. Do složky \texttt{js/lib} byly přesunuty všechny zdrojové kódy z knihovny Guacamole-js a byla vytvořena nová stránka \emph{/aspi} poskytující stejnou funkcionalitu jako \emph{JAVA} modul. Spolu s JavaScriptovým kódem bylo nutné přesunout i základní \gls{css} soubory pro zachování formátu vykreslování okna vzdálené plochy. Jak knihovny, tak styly jsou importovány pouze na stránce \emph{/aspi} pro snížení náročnosti běhu celé platformy, která by jinak načítala nepotřebné soubory. 

\section{Implementace CORS}
Pro možnost komunikace mezi více doménami bylo potřeba implementovat \gls{cors} (viz. sekce \ref{sec:technologies}). Hlavní změna je u zpracování požadavků na straně serveru, k čemuž musela být do projektu přidána a nastavena knihovna k jeho zpracování. Pro \emph{JAVA EE} byla využita knihovna CORS filter od [d]zhuvinov  [s]oftware\footnote{http://software.dzhuvinov.com/cors-filter.html}, která tuto implementaci poskytuje ve formě knihovny. Na stránkách je i detailně popsáno nastavení a použití, kdy v připadě tohoto serveru bylo nutno omezit zdroj \gls{cors} požadavků na jedinou doménu a více nebylo nutno se o bezpečnost obávat.

Druhá část implementace je umožnění cross-domain požadavků v JavaScriptu. Knihovna Guacamole-js musela být lehce přepsána v místech, kde byla použita volání \texttt{XMLHttpRequest}, která jsou v tomto případě volána asynchronně. Použité řešení bohužel nelze aplikovat na \gls{ie}, neboť \texttt{XDomainRequest} nepodporuje asynchronní zpracování a bude nutné toto řešení ještě přepracovat do funkční podoby (změna není součástí této práce).

\section{Implementace úpravy nastavení z administrace portálu}
Guacamole typicky poskytuje nastavení pomocí konfiguračního souboru umístěného ve složce instalace, nebo v domácí složce uživatele spouštějícího program. Jelikož je ke změně těchto konfiguračních souborů nutné přistupovat přímo k serveru pomocí \gls{ssh}, byla administrační část přepracována a zpřístupněna přímo z administrace portálu. Nastavení je rozdělené na dvě stránky: 

\begin{description}
  \item[ESF Nastavení] \hfill \emph{/config/esf/settings} \\
     Zde je možné upravit základní nastavení projektu jako je \gls{url} adresa pro připojení ke \emph{Guacamole} \gls{servlet} službám či přístup ke konfiguračnímu souboru. Ten se typicky jmenuje \texttt{guacamole.properties} a nachází se na serveru ve složce \texttt{/srv/guacamole/}. Je nutné se ujistit, že jak \emph{Guacamole} \gls{servlet} (\emph{Apache TomCat}) tak \gls{esf} Portál (\emph{Apache}) mají k souboru přístup a mohou jej upravovat.
  \item[Nastavení vzdálené plochy] \hfill \emph{/config/esf/remote} \\
Na této stránce lze za předpokladu, že lze přistupovat ke konfiguračnímu souboru, upravovat nastavení \emph{Guacamole} a \gls{url} pro připojení k serveru vzdálené plochy - což je v našem případě \gls{aspi}. Dále je nutné nastavit port pro připojení ke \emph{Guacd} démonu. Uživatelské jméno a heslo je převzato z nastavení každého jednotlivého uživatele.
\end{description}
