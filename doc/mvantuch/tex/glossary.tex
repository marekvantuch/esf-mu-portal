\newacronym{ie}{IE}{Internet Explorer}

\newacronym{esf}{ESF}{Ekonomicko Správní Fakulta}

\newacronym{eu}{EU}{Evropská Unie}

\newglossaryentry{aspi}{
  name=ASPI, 
  description={,,Automatizovaný Systém Právních Informací'' - informační systém vyvíjený společností Wolters Kluwer, poskytující komplexní informace z právníckých oborů}
}

\newglossaryentry{api}{ 
  name=API, 
  description={Zkratka z anglického Application Programming Interface, definující rozhraní poskytované k integraci programů třetích stran}
}

\newglossaryentry{cms}{ 
  name=CMS, 
  description={Content Management System}
}

\newglossaryentry{css}{ 
  name=CSS, 
  description={}
}

\newglossaryentry{responsive}{
  name=responsivní web design, 
  description={Způsob stylování webových dokumentů, při kterém je brán ohled na různá rozlišení klientských zařízení (telefon, tablet, počítač)}
}

\newglossaryentry{wysiwyg}{
  name=WYSIWYG, 
  description={Zkratka výrazu ,,What you see is what you get'', doslowně přeloženo jako ,,dostaneš to co vidíš''. Používá se pro označení editorů html kódu, které poskytují formátování pomocí tlačítek a výstup automaticky konvertují do html kódu}
}

\newglossaryentry{url}{
  name=URL,
  description={}
}

\newglossaryentry{ruby} {
  name=Ruby, 
  description={}
}

\newglossaryentry{bundler} {
  name=Bundler, 
  description={}
}

\newglossaryentry{bash} {
  name=BASH, 
  description={}
}

\newglossaryentry{pear} {
  name=PEAR, 
  description={}
}

\newglossaryentry{curl} {
  name=CURL, 
  description={}
}

\newglossaryentry{composer} {
  name=Composer, 
  description={}
}

\newglossaryentry{xss} {
  name=XSS, 
  description={Cross Site Scripting, technika ...}
}

\newglossaryentry{session} {
  name=relace,
  description={Označuje přetrvávající spojení mezi serverem a klientem}
}
