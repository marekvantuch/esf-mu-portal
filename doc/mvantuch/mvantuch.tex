%% Requires fithesis2 module (can be downloaded from
%% https://github.com/arax/fithesis2
%% Load document class fithesis2
%% {10pt, 11pt, 12pt}
%% {draft, final}
%% {oneside, twoside}
%% {onecolumn, twocolumn}
%% sudo yum install texlive texlive-babel-czech texlive-hyphen-czech 
\documentclass[10pt,final,oneside]{fithesis2}

%% Basic packages
\usepackage[czech]{babel}
\usepackage{cmap}
\usepackage[T1]{fontenc}
\usepackage{lmodern}
\usepackage[utf8]{inputenc}
\usepackage{graphicx}

%% Additional packages for colors, advanced
%% formatting options, etc.
\usepackage{color}
\usepackage{microtype}
\usepackage{url}
\usepackage{cslatexquotes}
\usepackage{fancyvrb}
\usepackage[small,bf]{caption}
\usepackage[plainpages=false,pdfpagelabels,unicode]{hyperref}
\usepackage[all]{hypcap}
\usepackage{amssymb}

%% Fix long URLs in DVIs
\usepackage{ifpdf}

\ifpdf
\else
  \usepackage{breakurl}
\fi

%% Packages used to generate various lists
\usepackage{makeidx}
\makeindex

%% CHECK WHAT THIS IS
\usepackage[xindy]{glossaries}
\makeglossaries

%% Use STAR and CIRCLE signs for nested
%% itemized lists
\renewcommand{\labelitemii}{$\star$}
\renewcommand{\labelitemiii}{$\circ$}

%% Title page information
\thesistitle{Vytvoření specializovaného klienta pro podporu e-learningového portálu}
\thesissubtitle{Diplomová práce}
\thesisstudent{Marek Vantuch}
\thesiswoman{false} %% Important when using Slovak or Czech lang
\thesisfaculty{fi}  %% {fi, eco, law, sci, fsps, phil, ped, med, fss}
\thesislang{cs}     %% {en, sk, cs}
\thesisyear{2014}
\thesisadvisor{Ing. Leonard Walletzký, Ph.D}

%% Beginning of the document
\begin{document}

%% Front page with a logo and basic thesis information
\FrontMatter
\ThesisTitlePage

%% Thesis declaration (required)
\begin{ThesisDeclaration}
  \DeclarationText
  \AdvisorName
\end{ThesisDeclaration}

%%\chapter*{Zadání práce}
%%Na základě analýzy možných Open Source klientů, podporujících připojení ke vzdálené ploše (Remonte Desktop) vyberte nejvhodnější a ten dále upravte tak, aby sloužil ke spuštění vzdálené aplikace po kliknutí na speciální odkaz v prohlížeči (aplikace ASPI). V případě nutnosti vytvořte obslužný plug-in do prohlížeče, sloužící pro předávání parametrů vzdálené aplikaci. Dále připravte obslužný systém pro vkládání textů, obsahujících odkazy do aplikace ASPI a jejich editaci.


%%TODO Thanks (optional)
\begin{ThesisThanks}
Chtěl bych obzvláště poděkovat Ing. Leonardu Walletzkému, Ph.D za příležitost pracovat na tomto projektu. Děkuji všem svým spolupracovníkům za ohleduplnost a trpělivost nutnou s ohledem na ztížené podmínky, které byly způsobeny mou prací ze zahraničí.
\end{ThesisThanks}

%%TODO Abstract (required)
\begin{ThesisAbstract}
Tato práce popisuje implementaci internetového portálu pro fakultu Ekonomicko-správní Masarykovy Univerzity. Jsou zahrnuty dvě hlavní části, popis úprav existujícího řešení postaveného na open-source CMS systému Drupal a jeho propojení pomocí vzdálené plochy na servery ASPI. Při vývoji byl kladen důraz na udržitelnost řešení, jeho strukturalizaci. Ačkoliv Drupal typicky udržuje všechna nastavení v databázi, pomocí dostupných nástrojů byla tato exportována do PHP kódu a byl nastaven proces jejich přenesení do testovacího či produkčního prostředí. Pro ulehčení tohoto procesu byly vytvořeny skripty a ty byly propojeny s ,,Continous Integration'' serverem. Důležitá byla i uživatelská přívětivost a jednoduchost, kterých bylo dosaženo sjednocením platformy postavené na Drupalu a existující implementace vzdálené plochy využívající HTML5 a JavaScriptu.
\end{ThesisAbstract}

%% Keywords (required)git
\begin{ThesisKeyWords}
Guacamole, vzdálená plocha, ASPI, ESF, Drupal, Team City, Drush, Phing, CORS, WebSockets, JAVA, PHP, JavaScript
\end{ThesisKeyWords}

\newacronym{ie}{IE}{Internet Explorer}

\newacronym{esf}{ESF}{Ekonomicko Správní Fakulta}

\newacronym{eu}{EU}{Evropská Unie}

\newacronym{cors}{CORS}{Cross-origin Resource Sharing}

\newacronym{is}{IS}{Informační systém}

\newacronym{muni}{MUNI}{Masarykova Univerzita}

\newacronym{rdp}{RDP}{Remote Desktop Protocol}

\newglossaryentry{ajax}{
  name=AJAX, 
  description={Asynchronní JavaScript a XML označuje způsob vývoje aplikací pomocí asynchronní komunikace mezi prohlížečem a serverem}
}

\newglossaryentry{aspi}{
  name=ASPI, 
  description={,,Automatizovaný Systém Právních Informací'' označuje informační systém vyvíjený společností Wolters Kluwer, poskytující komplexní informace z právníckých oborů}
}

\newglossaryentry{api}{ 
  name=API, 
  description={Aplikační rozhraní (API z anglického \emph{Application Programming Interface}) označuje rozhraní poskytované k integraci programů třetích stran}
}

\newglossaryentry{bundler} {
  name=Bundler, 
  description={Nástroj pro jednoduchou správu a instalaci gemů (balíčků programů programovacího jazyka Ruby)}
}

\newglossaryentry{ci}{
  name=CI,
  description={Průběžná integrace (CI z anglického \emph{Continuous Integration}) označuje souhrn nástrojů použitých k průběžné kontrole zdrojového kódu. Typicky sem patří spouštění testů, kontrola kvality kódu, statická analýza kódu a podobně}
}

\newglossaryentry{cms}{ 
  name=CMS, 
  description={Systém pro správu obsahu (CMS z anglického \emph{Content Management System}) označuje typicky internetovou aplikaci umožňující uživatelům úpravu obsahu. Bývá také označován jako redakční systém}
}

\newglossaryentry{css}{ 
  name=CSS, 
  description={Kaskádové styly (CSS z anglického \emph{Cascading Style Sheets}) je jazyk určený k popisu vzhledu webových stránek}
}

\newglossaryentry{cvs}{
  name=CVS,
  description={Systém ke správě verzí projektu (CVS z anglického \emph{Concurrent Version System}) slouží k ukládání historie verzí zdrojového kódu}
}

\newglossaryentry{deployment}{
  name=nasazení,
  description={Proces instalace projektu na typicky vzdálený server a spuštění případných migračních skriptů a pododbně}
}

\newglossaryentry{framework}{
  name=framework,
  description={Označení pro nástroj ulehčující vývoj software, typicky obsahující podpůrné knihovny, nástroje či popisující správný postup vývoje}
}

\newglossaryentry{mediaqueries}{
  name=@Media-Queries,
  description={Pravidla jazyka CSS umožňující podmínit použití vnořených pravidel dle určíté podmínky (typicky rozlišení monitoru a podobně)}
}

\newglossaryentry{opensource}{
  name=open-source,
  description={Software jehož zdrojový kód je volně dostupný a dle licence i upravitelný}
}

\newglossaryentry{responsive}{
  name=responsivní web design, 
  description={Způsob stylování webových dokumentů, při kterém je brán ohled na různá rozlišení klientských zařízení (telefon, tablet, počítač)}
}

\newglossaryentry{servlet}{
  name=Servlet,
  description={Program v jazyce JAVA, který na straně serveru zpracovává HTTP požadavky}
}

\newglossaryentry{ssh}{
  name=SSH,
  description={Zabezpečený komunikační protokol (z anglického \emph{Secure Shell} používaný v TCP/IP sítích}
}

\newglossaryentry{wysiwyg}{
  name=WYSIWYG, 
  description={Zkratka anglického \emph{,,What you see is what you get''}, doslowně přeloženo jako ,,dostaneš to co vidíš''. Používá se pro označení editorů html kódu, které poskytují formátování pomocí tlačítek a výstup automaticky konvertují do html kódu}
}

\newglossaryentry{url}{
  name=URL,
  description={Jednotný lokátor zdrojů (URL z anglického \emph{Uniform Resource Locator} označuje řetězec znaků definující jedinečné umístění}
}

\newglossaryentry{ruby} {
  name=Ruby, 
  description={}
}

\newglossaryentry{session} {
  name=relace,
  description={Také jako sezení, označuje přetrvávající spojení mezi serverem a klientem}
}

\newglossaryentry{uco} {
  name=UČO, 
  description={Unikátní číslo studenta či zaměstnance vysoké školy}
}

\newglossaryentry{xss} {
  name=XSS, 
  description={Využití bezpečnostních chyb stránky (XSS z anglického \emph{Cross-Site Scripting} za pomoci narušení skriptů stránek a podstrčení změněného kódu či dat}
}


%% Beginning of the thesis itself
\MainMatter

%% TOC (required)
\tableofcontents

%% Thesis text structured using
%% chapters, sections, subsections, etc.
\chapter{Úvod}
Tato práce se zabývá projektem, jehož cílem je vytvoření informačního systému pro studenty Ekonomicko-správní fakulty Masarykovy Univerzity. Stávající řešení v podobě webové stránka bylo postaveno na \gls{opensource} projektu \emph{Drupal}\footnote{https://drupal.org} verze 6 a celá jeho stuktura byla řešena na bázi slovníků a stránek, bez rozlišení typů obsahu. Mgr. Ondřej Materna ve své práci zanalyzoval možnosti zlepšení řešení a reálné požadavky studentů. Výsledkem je návrh řešení, který se neslučoval s existujícími stránkami, které tak musely být razantně přepracovány. \emph{Drupal} verze 6 existoval již více než pět let a dva roky existuje i jeho aktualizovaná verze 7~\cite{website:wiki:drupal}. Ten poskytuje vyšší rychlost, stabilitu i rozšíření díky širší podpoře komunity a nezávislých vývojářů. Stávající řešení je hlouběji rozebráno v kapitole~\ref{chap:analyza}, ve které jsou zároveň popsány technologie využité k implementaci nové verze portálu a základní architektura řešení.

Důležitou částí portálu je propojení s \gls{aspi} za pomoci klienta vzdálené plochy. Ve stávajícím řešení byli studenti nuceni používat jednu ze dvou možností připojení:

\begin{enumerate}
  \item lokální instalace nativní aplikace \gls{aspi} na klientský počítač a  její spuštění - odkazy se otevírají přímo v aplikaci
  \item připojení se na vzdálenou plochu pomocí jednoho z veřejně dostupných klientů a používání nativní aplikace zde
\end{enumerate}

Odkazy na stránkách se však automaticky nepřenášely na vzdálenou plochu a celkově vyžadovalo toto řešení vyšší technickou zdatnost uživatelů.

Hlavním cílem projektu je uživatelská přívětivost a proto byl při výběru řešení kladen důraz převážně na jednoduchost a minimální požadavky na klientské zařízení a uživatele. Bylo zvoleno řešení postavené na prvcích jazyka HTML5 a open-source nástroji \emph{Guacamole}. Ten poskytuje klienta vzdálené plochy čistě skrze okno prohlížeče. Komunikace se vzdálenou plochou probíhá za pomoci sprostředkovatelského proxy serveru. Architektura je detailně popsána v kapitole \ref{chap:implementace-guacamole}.

Jak bylo zmíněno výše, tato práce staví na diplomové práci Mgr. Ondřeje Materny ,,Návrh a realizace právního portálu pro ESF MU'' \cite{omaterna2013}. Její obsah je důkladně zanalyzován a z ní vyplývající poznatky uplatněny na prostředí \gls{cms} \emph{Drupal} a jeho možnosti. Proces aktualizace a implementace funkční a základ vzhledové stránky řešení jsou popsány v kapitole  \ref{chap:implementace-drupal}. Vzhled samotný není předmětem této práce, nýbrž práce Bc. Ivany Haraslínové. 

Z důvodu spolupráce mezi více studenty byl vytvořen základní systém pro vedení projektu, využívající prostředí systému \emph{GitHub} a jeho stručný popis je obsažen v kapitole~\ref{chap:vyvoj}. Projekt je veřejně dostupný na adrese \url{https://github.com/kanei/esf-mu-portal}.

\chapter{Analýza}
\label{chap:analyza}
Projekt byl od začátku koncipován jako aktualizace stávajícícho řešení, do kterého budou promítnuty zkušenosti a znalosti nabyté za dobu jeho fungování, zkombinované s moderními technologiemi, které nebyly v době vývoje předešlé verze dostupné. Návrh klade důraz na minimalizaci možných uživatelských chyb a jednoduchost a názornost vytváření obsahu. Těchto cílů dosahuje za pomoci definování vztahů mezi obsahem a omezením možností úprav automatickým generováním prvků webu, jakými jsou například seznamy a tabulky zobrazující propojení elementů.

\section{Existující řešení}
Při analýze bylo nutné zohlednit stávající řešení a možnosti jeho rozšíření. Existující portál využíval \emph{Drupal} v 6. verzi, který je již značně zastaralý a jeho rozšiřitelnost omezená, či neadekvátně komplikovaná. Řešení bylo rozšířeno řadou volně dostupných modulů, které dohromady poskytovaly komplexní strukturu nabídky menu vytvořenou namíru požadavkům fakulty. Struktura byla navržena v době, kdy nebylo známo, že systém bude pro zobrazování zákonů využívat externí software ASPI, proto bylo počítáno s možností zobrazení zákonů přímo v systému. 

Autentizace na server probíhala pouze za pomocí jednotného hesla, které bylo měněno ve čtvrtletních cyklech. Heslo nebylo nijak vázáno na jednotlivé uživatele a tento způsob nedovoloval téměř žádnou kontrolu nad uživateli přistupujícími k obsahu - po získání hesla byl schopen na server přistupovat kdokoliv a zabránění přístupu bylo možné pouze na bázi IP adresy uživatele. 

\subsection*{Struktura menu}

Navigace systému se nachází v levé části stránky a obsahuje v jediné stromové struktuře veškeré odkazy na obsah portálu.

\begin{itemize}
  \item \textbf{[název předmětu 1]} \hfill \\
    rozbalí/sbalí menu
  \begin{itemize}
    \item \textbf{[název předmětu 1]} \hfill \\
      odkaz na stránku předmětu, obsahující pouze název a informace o spolufinancování \gls{eu}
    \item \textbf{Studijní text} \hfill \\
      rozbalí/sbalí menu
      \begin{itemize}
        \item \textbf{Studijní text} \hfill \\
          odkaz na stránku studijní text, která je typicky prázdná
        \item \textbf{Úvod} \hfill \\
          odkaz na stránku obsahující typicky úvodní text k danému předmětu
        \item \textbf{[studijní text 1]} \hfill \\
          jeden z odkazů na stránku s obsahem přednášek, či jinak tématicky oddělených informací k předmětu
        \item \textbf{[studijní text 2]}
      \end{itemize}
    \item \textbf{Ostatní studijní materiály} \hfill \\
      tento a následující již nejsou pevně dané a vyskytují se nepravidelně pouze u některých předmětů
    \item \textbf{Prezentace}
    \item \textbf{Právní předpisy}            
    \item \textbf{Judikáty}
    \item \textbf{Procvičování}        
  \end{itemize}
  \item \textbf{[název předmětu 2]}
  \item \textbf{[název předmětu 3]}
\end{itemize}

Z výše popsané struktury vyplývá, že její vlastnosti nesplňují požadavky moderních informačních systémů. Odkazy mají nekonzistentní chování - buď se rozbalí pod-menu s dalšími odkazy, nebo prohlížeč přejde na novou stránku. Uživatel proto nemůže jednoduše vědět, jaké chování od své akce očekávat. Struktura obsahuje duplicity (název předmětu a odkaz na studijní text se opakují a v obou případech mají jiný význam). Dále není určená společná struktura všech předmětů a tím může lehce dojít ke zmatení uživatelů, kteří musejí pokaždé informace hledat na jiném místě (stránky \emph{Ostatní studijní materiály}, \emph{Prezentace}, \dots). Některé stránky také místo agregace užitečných informací obsahují buď pouze banner, nebo jsou úplně prázdné (úvodní stránka předmětu) v místech.

\section{Použité technologie}
\label{sec:technologies}

Základními technologiemi jsou \emph{Drupal} a \emph{Guacamole}, které definují vyžití dalších technologií. Základem jsou programovaí jazyky \emph{PHP}, \emph{JAVA} a \emph{JavaScript} a výstup je realizován pomocí \emph{HTML} (v.4 a v.5) a \emph{CSS} (v.2 a v.3). Ačkoliv bylo využito velké množství modulů, zmíněny budou pouze pouze ty, které měly zásadní vliv na funkcionalitu nebo proces vývoje.

\subsection{Jádro CMS a přidružené technologie}

\subsubsection*{\textbf{Drupal} \hfill \emph{http://drupal.org}} 
\label{subsec:drupal}
V době psaní této práce byl \emph{Drupal} světově třetím nejrozšířenějším \gls{cms}\cite{website:cms-market-share}. Založený na jazyce PHP, klade důraz na vývojáře a na možnosti úpravy stránek, proto často bývá označován za \gls{framework}. Konkurenční \gls{cms} \emph{WordPress} cílí na uživatelskou jednoduchost a většina stránek na něm postavných je lehce rozpoznatelná, zatímco \emph{Drupal} je možné změnit od základu a jeho \gls{api} pro tvorbu modulů poskytuje komplexní možnost úpravy. Minimálním požadavkem pro spuštění je \emph{PHP} verze 5.3 a i když některé jeho části jsou již implementovány objektově, celkově převládají čísté funkce s množstvím propriétárních principů řešení. Mezi ty patří například hook\_api, poskytující přípojné body pro další moduly a tím rozšiřitelnost částí systému, nebo systém vzorů (templates), které pomocí speciální jmenné konvence umožňují měnit výpis html kódu prvků webu (viz. kapitola \ref{sec:tema-vzhledu}).

\emph{Drupal} podporuje využití tzv. instalačních profilů. Jedná se o speciální moduly neposkytující samy o sobě žádnou funkcionalitu, ale sdružující informace o potřebných modulech a jejich iniciálním nastavení. To lze využít pro specifické distribuce, například vícejazyčná instalace, řešení pro blog, řešení orientované na výkon a podobně.

\subsubsection*{\textbf{Kerberos} (Protocol) \hfill \emph{http://web.mit.edu/kerberos}}
Vyvinut na \emph{Massachusettském technologickém institutu (MIT)}, protokol \emph{Kerberos} poskytuje autentizační funkcionalitu za pomocí symetrické kryptografie (jediný klíč je použit pro šifrování i dešifrování). Autentizace je provedena za pomoci důvěryhodného třetího serveru (v tomto případě \gls{is} \gls{muni}). %TODO Opravit na Masarykovy Univerzity

\subsubsection*{\textbf{Omega} (Drupal Theme) \hfill \emph{http://drupal.org/project/omega}}
\label{subsec:omega}
Pro \emph{Drupal} existuje nespočet témat vzhledu, které se starají o formátování výstupu html kódu a s ním spojených kaskádových stylů \gls{css}. Projekt \emph{Omega} je zaměřen na \gls{responsive} a poskytuje implementaci stylů pro různá zobrazovací zařízení. Zatímco třetí verze obsahuje komplexní administrační rozhraní, ve kterém lze nastavit rozmístění a poměry mezi bloky stránky, přístup čtvrté verze se zaměřil na implementaci v kódu. Poměry mezi jednotlivými bloky jsou zapsány pomocí nástroje \emph{SUSY} a pomocí knihovny \emph{Breakpoints}\footnote{http://xoxco.com/projects/code/breakpoints/} a vlastnosti \gls{mediaqueries} se mezi sebou přepínají.

\subsubsection*{\textbf{SASS} (Syntactically Awesome Style Sheets) \hfill \emph{http://sass-lang.com/}}
\label{subsec:sass}
Díky tématu vzhledu \emph{Omega} se nabídla možnost využití nástroje \emph{SASS} pro generování popis vzhledu \emph{HTML} elementů. Namísto přímého použití \gls{css} je kód napsán ve formátu \emph{SASS} a poté zkompilován do \gls{css}. Tento přístup přináší bezproblémovou kompatibilitu se všemi prohlížeči doplněnou o rozšířené možnosti definice pravidel, jako je využití proměnných, vnoření pravidel, in-line import a další. Kromě výrazného vylepšení čitelnosti a možnosti seskupováni pravidel bez zvyšování zátěže na přenos dat tak lehce lze dosáhnout i snížení programátorské náročnosti a zvýšení efektivity.

\subsubsection*{\textbf{COMPASS} \hfill \emph{http://compass-style.org/}}
\label{subsec:compass}
Soubory \emph{SASS} musí být kompilovány do \gls{css} pro zobrazení prohlížečem a protože manuální kompilace vyžaduje opakované spouštění příkazů a je tím pádem náchylná k opomenutí, nabízí se využití programu \emph{COMPASS}. Přestože poskytuje širokou škálu funkcionality určené pro ulehčení práce designérům, pro potřeby tohoto projektu byla využita pouze automatická konverze z \emph{SASS} do \gls{css} dle definovaných pravidel za pomocí příkazu \texttt{\$ compass watch}. Po spuštění démona jsou kontrolovány všechny v souborech ve složce a automaticky regenerovány výstupní \gls{css} soubory pro načtení prohlížečem.

\subsubsection*{\textbf{SUSY} \hfill \emph{http://susy.oddbird.net/}}
\label{subsec:susy}
Další technologií využitou v tématu vzhledu je \emph{SUSY} - implementace responsivní mřížky pro \emph{COMPASS}. Za pomocí jednoduchých pravidel lze nadefinovat rozdílné rozvržení stránky v závislosti na rozlišení zobrazovacího zařízení. Například monitoru počítače s rozlišením vyšším než 1024 bodů můžeme postranní panel zobrazit nalevo, zatímco na mobilním zařízením můžeme text zmenšit a zobrazit v horní části stránky spolu s vypuštěním některých nedůležitých bloků. Celá stránka je může být rozdělena na počet sloupců, které mohou být dynamicky vyplňovány a pravidla lze velmi jednoduše zapisovat bez nutnosti řešení problémů s kompatibilitou mezi prohlížeči.

\subsection{Technologie připojení ke vzdálené ploše}

\subsubsection*{\textbf{Guacamole} \hfill \emph{http://guac-dev.org/}}
\label{subsec:guacamole}
Jak je popsáno na stránkách této knihovny, jedná se o potvrzení myšlenky připojení ke vzdálené ploše skrze webový prohlížeč. Namísto potřeby instalace speciálního klienta je možné používat jakýkoliv prohlížeč podporující některé z technologií \emph{HTML5} (např. canvas). Připojení probíhá skrze proxy server implementovaný v jazyce \emph{C}, který zprostředkovává připojení a komunikaci. Ke klientovi putují již jen obrazová data a povely, obojí zakódováno v proprietárním protokolu, který umožňuje snížit odezvu obrazu a dekódovat jej v průběhu přijímání dat. Postup komunikace mezi klientským prohlížečem a vzdálenou plochou je blíže vyobrazen na diagramu \ref{fig:arch_core}.

\begin{figure}[htp] 
  \centering{\includegraphics[width=12cm]{img/architecture-core-crop.pdf}}
  \caption{Architektura komunikace nástroje Guacamole skrze vzdálenou plochu}
  \label{fig:arch_core}
\end{figure}  

Propojení mezi webovým prohlížečem a proxy serverem je provedeno za pomocí \gls{ajax} požadavků v čistém jazyce \emph{JavaScript}, zatímco propojení z proxy dále podporuje protokoly \emph{SSH}, \emph{VNC} a \emph{RDP}.

\subsubsection*{\textbf{CORS} \hfill \emph{http://www.w3.org/TR/cors/}} 
Kvůli bezpečnostním rizikům je \emph{JavaScriptu} běžícímu v prohlížeči zakázáno přistupovat k jiným doménám, tento přístup se nazývá \emph{Same-origin policy} a slouží k zamezení podvodných stránek a zvýšení internetové bezpečnosti. Veškeré požadavky směřující na jinou doménu tak musejí být vykonávány přímo ze serveru a prohlížeči zasílány pomocí protokolu \gls{ajax}. Tento přístup je často zbytečně náročný a v dnešní době bývá nutné tato omezení obejít. K tomu slouží \gls{cors}~\cite{website:cors}, kdy za pomocí \emph{HTML} hlaviček přidaných do komumnikace mezi serverem a prohlížečem a nastavení serveru je umožněna komunikace s jinými doménami. Pokud server odešle zpět hlavičku \texttt{Access-Control-Allow-Origin: [doména]}, prohlížeč ji zanalyzuje a případně umožní komunikaci. K tomuto je nutné využít \emph{XMLHttpRequest} (\emph{XDomainRequest} v případě \gls{ie}) s attributem \texttt{WithCredentials}, o zbytek se postará prohlížeč se serverem a komunikace funguje stejně jako v případě klasických požadavků.

\subsubsection*{\textbf{ASPI} \hfill \emph{http://www.systemaspi.cz/}}
Systém \emph{ASPI} poskytuje jeho uživatelům komplexní právní informace z prostředí České republiky. Velký důraz je kladen na možnost jednoduchého vyhledávání a verzování obsahu, stejně jako dostupnost skrze hlavní operační systémy. \emph{ASPI} funguje jako aplikace instalovaná na klientském prostředí, ale nabízí také možnost vytvoření serverů dostupných k připojení skrze vzdálenou plochu, čehož je využito v případě tohoto projektu.

\subsection{Podpora nasazení a vývoje projektu}

\subsubsection*{\textbf{Drush} \hfill \emph{http://github.com/drush-ops/drush}}
\label{subsec:drush}
Pro ulehčení administračních úkonů nad instalací \emph{Drupalu} byl komunitou vyvinut program \emph{Drush}, poskytující textové administrační rozhraní nad Drupalem v terminálové konzoli. V základu jsou umožněny operace nad moduly a tématy vzhledu a každý z modulů může pomocí implementace funkcí v souboru \texttt{[název].drush.inc} zoršířit rozhraní vlastními příkazy.

\subsubsection*{\textbf{Phing} \hfill \emph{http://www.phing.info/}}
\label{subsec:phing}
\emph{Phing} je nástroj určený ulehčení \gls{deployment} a nastavení \emph{PHP} projektů, fungující na bázi \emph{XML} konfiguračního souboru. Je založený na podobné myšlence jako \emph{Apache ANT}, který poskytuje podobnou funkcionalitu pro jazyk \emph{JAVA}. V konfiguračním souboru lze definovat pravidla, proměnné a závislosti vedoucí k \gls{deployment} projektu, kontrolám kódu, či automatizaci jakýchkoliv dalších akcí, potřebných k správnému chodu aplikace.

\subsubsection*{\textbf{DrushTask} \hfill \emph{https://drupal.org/project/phingdrushtask}}
Tento projekt využívá možnosti vývoje vlastních operací (task) v nástroji \emph{Phing}. Za pomoci příkazu \texttt{TaskdefTask}\cite{website:phing-user-guide} je definovaná operace \texttt{<drush>}, poskytující rozhraní k programu \emph{Drush}. Tím je umožněno automatizovat většinu administrátorských činností nad projektem a případně implementovat automatickou analýzu kódu a testování pomocí \gls{ci}.

\subsubsection*{\textbf{Maven} \hfill \emph{http://maven.apache.org/}}
Podobně jako \emph{Phing} pro {PHP}, \emph{Maven} poskytuje automatizaci \gls{deployment} \emph{JAVA} projektů (převážně). Jako základ také používá \emph{Apache ANT}, který rozšiřuje a specializuje se na možnost znovuvyužití skriptů pro více projektů. 

\subsubsection*{\textbf{Team City} \hfill \emph{http://www.jetbrains.com/teamcity/}}
Podobně jako \gls{opensource} komunitou často využívaný \emph{Jenkins} či \emph{Hudson}, \emph{Team City} je implementaci \gls{ci} serveru. V případě tohoto řešení jde o čistě komerční produkt, který je ale pro malé projekty dostupný zcela zdarma (je omezen na počet projektů). \emph{Team City} umožňuje komplexní nastavení projektů a jednotlivých úkonů, kontrolujících zdrojový kód, nebo provádějících operace \gls{deployment} na produkční prostředí. Je také poskytována integrace s většinou moderních \gls{cvs}.

\subsubsection*{\textbf{GitHub} \hfill \emph{https://github.com/}}
\label{subsec:github}
Pro udržování historie kódu a pro spolupráci více autorů byla vybrána \gls{cvs} technologie \emph{Git} a online portál \emph{GitHub}, který poskytuje jednoduchou instalaci a online správu uložiště. Všechny úpravy mohou být jednoduše zobrazeny přímo v prohlížečí a provázány s úkoly vytvořenými v úkolovém managementu na stejném internetovém portálu. 

\section{Struktura portálu v prostředí Drupalu}

Při vývoji informačních systémů jsou typicky data ukládány v databázi. To stejné platí i pro Drupal, pouze s tím rozdílem, že poskytuje administrační rozhraní schopné dynamicky vytvářet úložiště pro data. Namísto vytváření nové tabulky jsou uživatelé schopni definovat typ obsahu, který navíc může dědit určité společné rysy dle jeho předka. Základním typem je Entita, poskytující možnost ukládání základních informací. Entita je však pouze abstraktní pohled na data a nemůže být instancializována, namísto toho musí nejdříve být definován typ entity, na jehož základě až pak může být vytvořen typ obsahu. Pro ilustraci, v jádře drupalu existuje typ entity Uzel (Node), definující obsah s vlastní \gls{url} adresou a titulkem. Ten však nemůže být vytvořen přímo a administrátor musí definovat typ obsahu (Bundle), například Stránka (Page), ke které je již možné přistupovat. Stránka je tedy uzlem a entitou. Toto však neplatí pro všechny typy obsahu, kdy soubory (File) a slovníky (Vocabulary) lze vytvářet přímo a nelze definovat jejich potomky\cite{drupal-entities}. Seznam všech základních entit je zobrazen v tabulce~\ref{tab:typy-entit}.

\begin{table}
  \caption{Základní typy entit v Drupalu}
  \label{tab:typy-entit}
  \begin{tabular}{ | p{3cm} | l | c | c | }
    \hline 
    Typ entity & Strojový název & Dostupnost polí & Rozšiřitelnost \\ \hline 
    Komentář & comment & \checkmark & \checkmark \\ \hline 
    Soubor & file &  & \\ \hline 
    Slovník & vocabulary &  & \\ \hline 
    Uzel & node & \checkmark & \checkmark \\ \hline 
    Uživatel & user & \checkmark & \checkmark \\ \hline 
    Záznam slovníku & term & \checkmark & \checkmark \\ \hline             
  \end{tabular}
\end{table}

Pokud daný typ entity poskytuje možnost ukládání polí, je možné mu přiřadit pole, které si lze předststavit v analogyi databáze jako sloupce tabulky. Tato pole jsou však reprezentována jako tabulky v databázi a tak je při přístupu k jedné entitě v Drupalu ve skutečnosti nutno přistoupit do tolika tabulek, kolik má daná entita přiřazených polí. Zároveň jdou pole využívat skrze více typů entit a tak mohou být pole sdílena skrze celý systém, například typ obsahu \emph{Stránka} a typ obsahu \emph{Novinka} mohou oba poskytovat pole \emph{Obrázek}, které bude v databázi ukládáno do jediné tabulky \emph{field\_obrazek}. Instance obou typů obsahu dostanou přiděleno unikátní číslo v rámci všech entit, které určuje její data v databázi. Ne všechna pole musejí být zobrazena na stránce entity, nebo může pro jejich zobrazení být využito speciálních formátovacích technik, jako tabulek, či jiných výpisů. Všechna tato nastavení lze opět definovat buď v kódu, nebo skrze administrační rozhraní \emph{Drupalu}.

\subsection{Typy polí použitelné pro vlastnosti typů obsahu}
V drupalu existuje několik základních typů polí, které fungují podobně jako datové typy v typických programovacích jazycích. Narozdíl od jednoduchých datových typů je ke každému navíc přiřazeno jedno či více formátování výsledného \emph{HTML} kódu a zpravidla poskytují rozšiřující nastavení. Formátování jde dále změnit pomocí přepsání základních formátovácích funkcí, či vytvoření nových.

\begin{description}
  \item[Text] Poskytuje pole pro text s omezenou délkou bez možnosti vkládání html značek. Jeho obsah je zadáván pouze v jediném řádku a poskytuje pouze limitované možnosti nastavení, kterými je délka vstupního pole, či defaultní hodnota. 
  
  \item[Dlouhý text se souhrnem] Narozdíl od typu pole Text může v tomto případě být text neomezeně dlouhý a obsahovat značky html kódu. Ty mohou být vkládány buď ručně a nebo automaticky formátovány pomocí \gls{wysiwyg} editoru. Zároveň je pro text možné vyplnit souhrn, který je ve specifických případech zobrazit, nebo se automaticky zobrazí zkrácená verze textu. Počet znaků je možné určit pro každou instanci pole zvlášť.
  
  \item[Hodnota slovníku] V nastavení pole lze vybrat jeden z existujících slovníků. Jeho hodnoty pak jsou vypsány v jednom z dostupných formátů (radio/zaškrtávací tlačítka) a dle nastavení pak lze vybrat jen jedna, nebo více možností.
  
  \item[Logická hodnota] Poskytuje pole pro logickou hodnotu - ano nebo ne. Pro hodnoty lze vybrat jednu z typicky použítelných nápisů (ano/ne, pravda/nepravda, checkmark/x), nebo definovat nápisy vlastní. 

  \item[URL adresa] Kromě samotné adresy poskytuje možnost uložení dalších parametrů, podobně jako element <a> jazyka html. Mezi tyto parametry patří např. titulek, cíl a podobně.

  \item[Média] Pro vkládání médií do systému je využit modul Media, který přidává komplexní řešení pro vkládání jakéhokoliv obsahu. Je možné vkládat dokumenty, soubory pdf, zvuk, obraz, či video a všechny tyto typy obsahu jsou uživately poskytovány v jednotném formátu.
  
  \item[Váha] I když by se pro určení váhy produktu dalo využít jednoduchého číselného pole, jeden z rozšiřujících modulů poskytuje jednak speciálně využitelné pole pro výběr hodnoty a také pohledy na obsah a řadící funkce, které celou implementaci značně ulehčují.
  
  \item[Odkaz na entitu] Důležitou vlastností většiny systémů je možnost propojení entit. Pro modelování těchto vztahů zde existuje typ pole odkaz na entitu, u kterého je možné vybrat typy entity, na které je daný typ obsahu navázán a způsob propojení. Za pomocí extra modulů lze také možné modelovat místo jednoduché asociace kompozici. V tom případě při smazání nadentity jsou smazány i všechny podentity. Lze také definovat, zda je možné pouze vytvářet nové podentity, či přidávat ty již existující.
\end{description}


\chapter{Implementace portálu v Drupalu}
\label{chap:implementace-drupal}

Tady popsat co přesně bylo uděláno v rámci implementace a proč to bylo uděláno. Proč bylo důležité aktualizovat na Drupal 7 a co to s sebou přineslo za změny.
\section{Aktualizace Drupal 6 na Drupal 7}
Výhody Drupal 7 nad Drupalem 6
Porovnání dostupných verzí mezi Drupalem 6 a 7

\begin{table}
  \caption{Porovnání verzí modulů mezi Drupalem 6 a 7}
  \begin{tabular}{ | p{5cm} | p{2.5cm} | p{2.5cm} | c | }
    \hline 
    Jméno modulu & Drupal 6 & Drupal 7 & stav  \\ \hline 
    Backup and Migrate & 2.7 & 2.7 & \checkmark \\ \hline
    Colorbox & 1.6 & 2.4 & \checkmark \\ \hline
    CCK & 2.9 & jádro & \checkmark \\ \hline
    Custom Breadcumbs & 1.5 & 1.x-alpha3 & \\ \hline
    DB Maintenance & 1.4 & 1.1 & \checkmark \\ \hline
    DHTML Menu & 3.5 & 1.0-beta1 & \\ \hline 
    Email Field & 1.4 & 1.2 & \checkmark \\ \hline
    File (Field Paths & 1.5 & 1.0-beta4 & \\ \hline
    FileField & 3.11 & jádro & \checkmark \\ \hline
    Front Page & 1.3 & 2.4 & \checkmark \\ \hline
    ImageAPI & 3.11 & jádro & \checkmark \\ \hline
    ImageCache & 2.0-rc1 & jádro & \checkmark \\ \hline
    IMCE & 2.5 & 1.7 & \checkmark \\ \hline
    IMCE Wysiwyg bridge & 1.1 & 1.0 & \checkmark \\ \hline
    jCarousel & 2.6 & 2.6 & \checkmark \\ \hline
    Link & 2.10 & 1.1 & \checkmark \\ \hline
    Localization Update & 2.10 & 1.1 & \checkmark \\ \hline
    Menu Attributes & 2.0-beta1 & 1.0-rc2 & \checkmark \\ \hline
    Menu Block & 2.4 & 2.3 & \checkmark \\ \hline
    Pathauto & 1.6 & 1.2 & \checkmark \\ \hline
    Taxonomy Breadcrumb & 1.1 & - & \\ \hline
    Token & 1.19 & 1.5 & \checkmark \\ \hline
    Transliteration & 3.1 & 3.1 & \checkmark \\ \hline
    Views & 2.16 & 3.7 & \\ \hline
    Views Search & 1.0 & - & \\ \hline
    WYSIWYG & 2.4 & 2.2 & \checkmark \\ \hline
  \end{tabular}
\end{table}

\subsection{Odebrané moduly}

\subsection{Nově přidané moduly a jejich popis}
Zde sepsat které moduly byly přidány a proč

\section{Téma vzhledu a jeho rozložení}

\section{Deployment a jeho možnosti}

\section{Optimalizace řešení a její dopad na uživatelskou zkušenost}
Ideálně si projít nějaké výzkumy Googlu ohledně odezvy a jejího dopadu na uživatele
Popsat agregaci JS, CSS a podobne 
\cite{website:drupal:optimizing}

\chapter{Implementace připojení ke vzdálené ploše}
\label{chap:implementace-guacamole}
Implementace připojení ke vzdálené ploše probíhala v několika fázích. Nejdříve bylo potřeba spustit řešení Guacamole lokálně a ujistit se, že je použitelné pro naše potřeby. Bylo také nutné analyzovat, jaké všechny možnosti poskytuje a které vlastnosti jsou potřebné a které přebytečné. Jako přebytečné byly idnetifikovány vlastnosti poskytující správu uživatelů, připojení a jejich skupin. Dále je implementována podpora klávesnici na obrazovce, ukládání snímků obrazovky a mnoho dalších rozšířených možností, které nebyly do základní verze vyžadovány a byly spíše kontraproduktivní s ohledem na zvýšení komplexnosti kódu a debugování celého řešení. 

Jakmile byla funkčnost ověřena, bylo nutné celou klientskou část řešení přesunout do modulu v Drupalu a upravit pro tamější podmínky. Bylo nutné změnit HTML výstup, kdy Drupal si v základu buduje strukturu pomocí několika šablonových souborů pro stránku, tělo a pod. Bylo nutné přidat JavaScriptové knihovny z Guacamole, odstranit nepotřebný kód a upravit vše k funkčnosti v prostředí s použitím CORS. 

Guacamole také poskytuje mnohá nastavení a pro ulehčení byla do modulu přidána administrační stránka, která přímo zapisuje do konfiguračního souboru Guacamole a tím eliminuje potřebu připojení na server. 

\section{Ostranění autentizace uživatelů a vytvoření vlastní autorizace}
Guacamole v základu poskytuje API pro správu uživatelských účtů a jejich připojení. Jsou implementovány dvě metody, jedna spoléhá na uložení dat do konfiguračního souboru, zatímco druhá pracuje s MySQL databází. Obě rozšiřují třídu \emph{SimpleAuthenticationProvider}, ke které přidávají potřebnou funkcionalitu. 

Pro potřeby tohoto projektu však bylo potřeba vždy pouze jediné připojení, pro jednotlivé uživatele se měnilo pouze jejich přihlašovací jméno. Byla vytvořena nová třída \emph{DrupalAuthenticationProvider}, která načítá adresu serveru a port přímo z konfiguračního souboru pro všechna připojení a vrací je Guacamole k dalšímu zpracování. 

Pro nastavení přihlašovacích údajů je potřeba se nejdříve k serveru přihlásit. Pro minimalizaci úkonů potřebných ke každému připojení ke vzdálené ploše a také k zjednodušení celého procesu si musí každý uživatel portálu uložit své přihlašovací údaje do svého profilu, odkud jsou pak použity při přihlášení ke Guacamole Servletu. K tomu je použito dvou služeb - \emph{connect} a {login}. Proces je znázorněn na diagramu \ref{fig:login_process}. 

Komunikace je započata přímo v klientském prohlížeči z JavaScriptu. Po přístupu na stránku /aspi je nejdříve kontaktována pomocí technolie AJAX stránka /aspi/ajax na portálu, která se připojí ke servletu Login. Guacamole Servlet si vytvoří nové \gls{session} a jeho identifikátor odešle zpět PHP kódu. Pokud vše proběhne v pořádku, je identifikátor sezení odeslán pomocí set-cookie hlavičky zpět stránce /aspi, kde si informaci uloží prohlížeč jako cookie pro pozdější komunikaci. V případě jakékoliv chyby je chybové hlášení vypsáno na obrazovce a zalogováno v systému bez jakýchkoliv dalších pokusů o připojení, uživatel je tedy nucen buď stránku obnovit, nebo kontaktovat správce stránek. Pokud však vše proběhne v pořádku, je inicializována smyčka přímých volání služby \emph{tunnel} poskytované Guacamole Servlet kontejnerem za pomocí technologie CORS. Komunikace probíhá za pomocí Guacamole protokolu a ten je poté konvertován na grafický výstup na obrazovce uživatele, který je od této chvíle schopen pomocí myši a klávesnice ovládat vzdálenou plochu a tím i ASPI.

\begin{figure}[]
  \includegraphics[scale=0.85]{img/login-process-crop.pdf}
  \caption{Proces připojení ke vzdálené ploše}
  \label{fig:login_process}
\end{figure}  

\section{Přesun JavaScript kódu a stylů do modulu Drupalu}
Pro ulehčení správy \gls{session}, které by jinak muselo být synchronizováno mezi dvěma stránkami, byl veškerý kód stránky z Guacamole JAVA projektu přesunut do modulu v drupalu. Do složky js/lib byly přesunuty všechny zdrojové kódy z knihovny Guacamole-js a byla vytvořena nová stránka /aspi poskytující stejnou funkcionalitu jako JAVA modul. Spolu s javascriptem bylo nutné přesunout i základní css styly pro zachování formátu vykreslování okna vzdálené plochy. Jak knihovny, tak styly jsou importovány pouze na stránce /aspi pro snížení náročnosti běhu celé platformy, která by jinak byla nucena načítat nepotřebné soubory. 

Drupal v základu vytváří komplexní html výstup včetně loga, základní struktury a podobně, kdežto guacamole vyžaduje pro vykreslení vzdálené plochy velmi jednoduchou strukturu obsahující prakticky jen plátno pro vykreslování výstupu a nastavení v hlavičce. Zatímco údaje do hlavičky lze přidat jednoduše pomocí implementace hook\_api drupalu (viz. sekce \ref{sec:technologies}), změnu výstupu bylo nutné implementovat pomocí kombinace několika technik. Téma vzhledu \emph{Omega} ve verzi 4.x poskytuje funkcionalitu rozdílných rozvržení (layouts) dostupných pro různé části stránky. Tato možnost lze skombinovat s modulem \emph{Context} a jeho podmodulem \emph{Context Omega}, který poskytuje přemostění mezi modulem a tématem vzhledu a tím i možnost změny rozvržení podle definovaných pravidel. V případě tohoto projektu stačilo nadefinovat pravidlo změny pro url /aspi na které se automaticky přepne rozvržení na minimalistické rozvržení nazvané \emph{Guacamole}. Toto nastavení je i exportováno do konfiguračního modulu esf\_feature.

\section{Implementace CORS}
Pro možnost komunikace mezi více doménami bylo potřeba implementovat CORS (viz. sekce \ref{sec:technologies}). Hlavní změna je u zpracování požadavků na straně serveru, k čemuž musela být do projektu přidána a nastavena knihovna k jeho zpracování. Pro JAVA EE byla využita knihovna CORS filter od [d]zhuvinov  [s]oftware\cite{website:cors-filter}, která tuto implementaci poskytuje ve formě knihovny. Na stránkách je i detailně popsáno nastavení a použití, kdy v připadě tohoto serveru bylo nutno omezit zdroj CORS požadavků na jedinou doménu a více nebylo nutno se o bezpečnost obávat.

Druhá část implementace je umožnění cross-domain požadavků v JavaScriptu. Knihovna Guacamole-js musela být lehce přepsána v místech, kde byla použita volání XMLHttpRequest, která jsou v tomto případě volána asynchronně. Použité řešení bohužel nelze aplikovat na \gls{ie}, neboť XDomainRequest nepodporuje asynchronní zpracování a bude nutné toto řešení ještě přepracovat do funkční podoby.

\section{Implementace úpravy nastavení z administrace portálu}
Guacamole typicky poskytuje nastavení pomocí konfiguračního souboru umístěného buď ve složce instalace, nebo v domácí složce uživatele, který program spouští. Jelikož je ke změně těchto konfiguračních souborů nutné přistupovat přímo k serveru, rozhodl jsem se tuto administrační část přepracovat a zpřístupnit přímo z administrace portálu. Ta nabízí dvě základní stránky. Na stránce Nastavení (/config/esf/settings) je možné upravit základní nastavení projektu jako je URL adresa pro připojení ke Guacamole Servlet službám či právě přístup ke konfiguračnímu souboru. Ten se typicky jmenuje guacamole.properties a nachází se ve složce /srv/guacamole/. Je nutné se ujistit, že jak Guacamole Servlet (TomCat) tak ESF Portál (Apache) mají k souboru přístup a mohou jej upravovat.

Na stránce nastavení Připojení ke vzdálené ploše (config/esf/remote) lze za předpokladu, že je vše správně nastaveno, upravovat nastavení Guacamole samotného a url pro připojení k serveru vzdálené plochy - což je v našem případě ASPI. Dále je nutné nastavit port pro připojení ke Guacd démonovi poskytujícímu proxy pro připojení ke vzdálené ploše, uživatelské jméno a heslo je však převzato z nastavení každého jednotlivého uživatele.

\section{Instalace a konfigurace řešení}
Instalace je možná prakticky "na zelené louce" za použití programů Phing a Drush. Na stroji je nutné mít nainstalovaný internetový server (Apache, Lighttpd, ...) s podporou PHP, SQL potřebné není - lze využít minimalistickou implementaci SQLite, která data ukládá do jediného souboru ve složce Drupalu. Drupal také k instalaci vyžaduje knihovny php-gd a php-pdo.

Před samotnou instalací je nutné nastavit základní vlastnosti projektu. Je nutné buď překopírovat soubor \texttt{phing/default.build.properties} do hlavní složky a přejmenovat jej na build.properties, nebo spustit příkaz \texttt{\$~phing}, který provede úvodní inicializaci, jejíž součástí je i překopírování souboru v případě, že neexistuje. V konfiguračním souboru je nutné v proměnné \texttt{project.dir} nastavit cílovou složku pro instalaci. Složka však nesmí existovat a uživatel spouštějící skript musí mít práva k jejímu vytvoření, protože jinak Drush není schopen s instalací pokračovat a skončí s chybovou hláškou. V konfiguračním souboru lze změnit další nastavení, jakými jsou administrátorský účet, url databáze a pod, ale pro základní funkcionalitu je možné tyto hodnoty ponechat nezměněné.

Jakmile je vše připraveno, lze začít s instalací, ta se provádí pomocí příkazu \texttt{\$~phing install} a v jejím průběhu se provedou následující kroky:

\begin{enumerate}
  \item kontrola - Je zkontrolována cílová složka a uživatel je případně upozorněn na nutnost jejího smazání.
  \item \emph{drush make} - Vytvoří se stuktura stránek, při čemž se ze souboru \texttt{esf.make} (který načte podsoubor \texttt{src/profiles/esf\_profile/esf\_profile.make}) načte seznam všech modulů využívaných v řešení spolu s jejich požadovanou verzí. Tyto se stáhnou do cílové složky, ve které se zároveň připraví konfigurační soubory pro instalaci Drupalu.
  \item překopírování proprietárních prvků webu - Do projektového adresáře jsou překopírovány moduly, témata a instalační profily projektu, aby s nimi mohl instalační skript pracovat.
  \item stažení knihoven - Při instalaci nejsou automaticky staženy některé knihovny potřebné k běhu stránek - jedná se o převážně o JavaScriptové knihovny umožňující správný grafický výstup témat vzhledu či administračního panelu. Jmenovitě se jedná o Backbone.js, Modernizr, html5shiv a underscore.
  \item \emph{drush site-install} - Spustí se instalace Drupalu z příkazové řádky, při které se povolí základní moduly a vytvoří databázová struktura. Je nainstalován instalační profil \emph{esf\_profile} a s ním jsou povoleny moduly rozšiřující funkcionalitu a také rysy dokončující nstavení stránek. Celý proces trvá až několik minut a v průběhu nijak uživatele neinformuje o průběhu, je tedy důležité vyčkat až do jeho ukončení.
\end{enumerate}

Po instalaci je vhodné zkontrolovat práva souborů a nastavit uživatele i skupinu na hodnoty daného internetového serveru (apache/lighttpd/...) za pomocí příkazů \texttt{\$~sudo~chown~apache~esf~-R} a \texttt{\$~sudo~chgrp~apache~esf~-R}. Toto je obzvláště důležité v případě použití SQLite databáze, ke které jinak nemá drupal práva zapisovat a tím pádem stránky spadnou s fatální chybou. 

Pokud vše proběhlo v pořádku a server je schopen přistupovat ke všem souborům i databázi, je možné na adrese propojené se stránkami navštívit úvodní stránku - typicky se jedná o \emph{http:\\localhost/esf}. Na stránky bylo převedeno veškeré nastavení, ale nebyl zde vytvořen žádný obsah. Ten lze buď migrovat z produkčních stránek, nebo generovat za pomocí modulu Devel. Na adrese \emph{http://localhost/esf/?q=user} se lze přihlásit za pomocí administrátorského účtu (pokud nebyl změněn v konfiguračním souboru, jedná se o admin/admin) a přistoupit k administračnímu rozhraní stránek.

\chapter{Organizace vývojového procesu}
\label{chap:vyvoj}

Veškerý vývoj probíhal na lokálně běžícím serveru. Pro minimalizaci nároků na výkon zařízení byl namísto typicky využívaného http serveru Apache nainstalován Lighttpd poskytující dostačující funkcionalitu při mnohem nižších požadavcích na výkon. Pro databázi byl namísto MySQL použit MySQL Lite, který nepotřebuje k běhu instalaci, ale celá implementace běží pouze nad jedním souborem uloženém na disku – v případě Drupalu mezi daty dané stránky. 

Pro dostupnost komplexního testování i bez využití internetu byl lokálně nainstalována služba Guacamole Proxy (Guacd) a místo připojení ke vzdálenému zařízeni byl virtuálně spuštěn systém Windows Server 2008 SP2, na kterém byla povolena vzdálená plocha pomocí protokolu RDP. Pro JAVA aplikaci byl použit server Apache Tomcat, stejně jako na produkčním serveru.

\begin{table}
  \caption{Porovnání technologií použitých na lokálním a produkčním prostředí}
  \begin{tabular}{ | p{3cm} | p{4cm} | p{4cm} | }
    \hline  
    & Vývojové prostředí & Produkční prostředí \\ \hline
    HTTP Server & Lighttpd & Apache \\ \hline
    SQL databáze & SQLite [OPRAVIT???] & MySQL \\ \hline
    JAVA Servlet Container & Apache Tomcat 7 & Apache Tomcat 7 \\ \hline
    Vzdálená plocha & Windows Server 2008 SP2 (Oracle VM VirtualBox) & ASPI [PŘIDAT VÍCE DETAILŮ] \\ \hline
  \end{tabular}
\end{table}

Protože Drupal je postaven z velké části na konfiguraci uložené v databázi, bylo potřeba vymyslet způsob, jak změny prováděné na lokálním stroji efektivně přenášet do produkčního prostředí. Pro uložení nastavení do konfiguračních souborů byl použit projekt Features [LINK], který exportuje pomocí funkcionality postkytnuté modulem Ctools nastavení drupalu jako jsou typy obsahu a také pomocí modulu Strongarm nastavení systéu samotného. Všechna tato nastavení jsou pak uložena do modulu Drupalu a mohou pak být jednoduše přenesena na jiné prostředí a také aktualizována v průběhu času a s přibývajícím vývojem. Pro udržení přehlednosti byla konfigurace rozdělena na tři části:

ESF Feature (esf\_feature) – základní nastavení systému včetně typů obsahů, vztahů mezi nimi, metody zadávání obsahu (WYSIWYG) a základní prvky zobrazené na stránkách
ESF Feature UI (esf\_feature\_ui) – administrační rozhraní pro správu obsahu a nádstavba nad modulem Workbench [LINK]
ESF Permissions (esf\_permissions) – definice uživatelských rolí a jejich práv
Moduly pak byly uloženy v repozitáři a tím zajištěno jejich správné verzování. V případě potřeby je možné je jednoduše doručit do produkčního prostředí a nastavení je pak možno načíst z daných modulů, kdy v případě bezproblémové aktualizace (v komponentech popsaných v daném modulu nebyly manuálně provedeny žádné změny) se změny provedou automaticky – jinak musí být přes rozhraní či drush manuálně určeno, která změna se má využít a zda se případně nemá aktualizovat daný modul.

Instalační skript a profil jsou generovány automaticky pomocí modulu Profiler\_Builder. Jeho výstupem je instalační soubor .make a instalační profil. V instalačním souboru .make je vypsán seznam modulů a jejich verzí, určený pro příkaz drush make, který stáhne potřebné projekty z repozitáře Drupalu. Po vytvoření je nutné ručně odstranit moduly esf\_*, které Drush neumí automaticky stáhnout a instalace by selhala. Díky rozdělení na dva hlavní a doplňkový instalační skript je možné definovat url adresy ke knihovnám, které nelze automaticky doplnit a ty zároveň nejsou přepsány opětovným vygenerováním, neboť hlavní instalační soubor je přesunut do kořenové složky a přejmenován na esf.make, který již není nutné aktualizovat. V instalačním profilu esf\_profile jsou obsažena základní nastavení portálu a seznam modulů, které je nutné povolit ke správné funkčnosti stránek. Ke každému modulu lze přiřadit i opravné balíčky (patch), které jsou buď automaticky dohledány na na stránkách Drupalu a jako odkazy přidány do profilu, nebo mohou být přidány dodatečně ručně. Při instalaci jsou automaticky aplikovány na kód. Díky propojení s features se na stránky automaticky dostane i rozšířené nastavení a struktura.

\section{Organizace řízení projektu}
S ohledem na nízký počet zainteresovaných osob a rozsah projektu jsme neimplementovali žádné pokročilé metody projektového řízení a rozhodli jsme se pro jednoduchý seznam úkolů. Jelikož jsme pro uložení zdrojových kódů projektu využili GitHub, bylo nejjednodušší jej rovněž využít pro správu úkolů. Ačkoliv se nemůže rovnat s platformami, specializujícími se jen daný úkol, poskytuje několik základních prvků - problém (issue), milník (milestone) a značku (tag). Značky lze jednoduše využít pro rozlišení mezi úkolem a chybou a také důležitostí problému. Milníky byly využity pro jednoduché plánování a sledování pokroku.

\subsection{Značky}
Typy úkolů
\begin{itemize}
  \item Úkol \emph{(task)} - úkol, který bylo třeba vykonat na projektu
  \item Chyba \emph{(bug)} - chyba nalezená na projektu, kterou bylo potřeba opravit
\end{itemize}
Priorita
\begin{itemize}
	\item Nízká \emph{(0-low)}
	\item Střední \emph{(1-medium)}
	\item Vysoká  \emph{(2-high)}
	\item Kritická \emph{(3-critical)}
\end{itemize}

\subsection{Milníky}
\begin{itemize}
  \item 0.1 | Inicializace - úvodní výzkum týkající se připojení ke vzdálené ploše a dostupných technologií
  \item 0.2 | Drupal modul - vytvoření modulu pro drupal a jeho základní funkcionalita
  \item 0.3 | Struktura a práva - struktura stránek, jejich obsahu a práva uživatelů k jejich použití
  \item 0.4 | Guacamole Drupal - přesun funkcionality z Guacamole do Drupal modulu a jeho propojení s JAVA server aplikací
  \item 0.5 | Test v praxi - změny potřebné k umístění řešení na produkční servery a vytvoření skriptů k automatizaci tohoto procesu
  \item 1.0 | Základní verze - spuštění základní funkční verze
  \item 1.1 | Produkční verze - vyřešení všech problémů, komunikace se stranou klienta a přípravy na reálné spuštění v produkčním prostředí
  \item 1.2 | Údržba - první z verzí, ve kterých se bude dodávat údržba řešení
\end{itemize}


%% Lists of tables and figures, glossary, etc.
\printindex
%%\printglossary
\listoffigures
\listoftables

%% Bibliography from references.bib
\begingroup
\def\tmpchapter{0}
\renewcommand{\chaptername}{}
\renewcommand{\thechapter}{}
%\addtocontents{toc}{\setcounter{tocdepth}{-1}}
\chapter{References}
\renewcommand{\chapter}[2]{}% for other classes

\bibliographystyle{plain}
\bibliography{mvantuch}

%\addtocontents{toc}{\setcounter{tocdepth}{2}}
\endgroup

%% Additional materials
\appendix

%% End of the whole document
\end{document}
